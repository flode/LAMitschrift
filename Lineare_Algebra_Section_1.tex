\hypertarget{contents}{}
\tableofcontents
\pagebreak
\phantomsection
\section*{Literatur}
\addcontentsline{toc}{section}{Literatur}
\underline{Mathematik für Informatiker: } Teschl, Hackenberger \\
\underline{Lineare Algebra: } Beutelspacher, Fischer, Lang (auf Englisch), Stammbach.
\section{Grundlegendes}
\subsection{Mengen}
\subsubsection{Definition (Relation)}
Gegeben sein Mengen $X$ und $Y$.  Eine Teilmenge des kartesisches Produkt $X\times Y=\{(x,y):x\in X, y\in Y\}$ heißt \underline{Relation} ($R$) zwischen $X$ und $Y$; im Fall $X=Y$ spricht man von einer Relation auf $X$.  Ferner: $R^{-1}_1=\{(y,x)\in Y \in X: (x,y)\in R\}$ heißt \underline{Umkehrrelation}.
\subsubsection{Beispiel}
Die Menge $R_0=\{(x,y)\in X\in Y: y \mathrm{\ ist\ Hauptstadt\ von\ } x$ ist eine Relation zwischen der Menge $X$ aller Länder und $Y$ aller Städte. \\
\begin{center}
\includegraphics[scale=0.4]{1-1-2.jpg}
\end{center}
\subsubsection{Beispiel}
Mit den Mengen $X=\mathbb{R}$ $Y=[0,\infty)$ ist $R_1=\{(x,\left| x\right|) \in X\times Y, X\in X\}$ ist eine Relation mit der Umkehrrelation $R^{-1}=\{(\left| x\right|,x):x\in X\}$.\\
\begin{center}
\includegraphics[scale=0.4]{1-1-3.jpg}
\end{center}
\subsubsection{Beispiel}
Mit den Mengen $X=Y=\mathbb{R}$ ist $R_2=\{(x,y)\in X\times Y:x \leq y\}$
eine Relation $R^{-1}_{2}={(y,x):x\leq y}$ \\
 \begin{center}
\includegraphics[scale=0.4]{1-1-4.jpg}
\end{center}
\subsubsection{Beispiel}
Die Menge $R_3=\{(x,y)\in C\times C:x\ \mathrm{und\ } y \mathrm{\ haben\ gleichen\ Hersteller}\}$ ist eine Relation auf der Menge aller Computer $C$.
\subsubsection{Beschreibung (Gerichtete Graphen)}
Relation $R$ auf endlichen Mengen $X$ können alternative wie folgt dargestellt werden.  Man repr\"{a}sen\-tiert die Elemente von $X$ als Punkte in der Ebene (\underline{Knoten}) und verbindet $x,y\in X$ genau dann durch einen Pfeil (\underline{gerichtete Kante}), wenn $(x,y)\in R$.  Das paar $(X,R)$ heißt \underline{gerichteter Graph} oder \underline{Digraph}, z.B. $X=\{a,b,c\}$ $R=\{(a,b),(b,c),(c,d)\}$.\\
\begin{center}
\includegraphics[scale=0.4]{1-1-6.jpg}
\end{center}
$X=\{a,b,c\}$ $R=\{(b,a),(a,a),(c,c)\}$. \\
Eine Relation $R$ auf X heißt \\
\underline{reflexiv} $\Leftrightarrow (x,x)\in R$ für alle $x\in X$ \\
\underline{transitiv} $\Leftrightarrow (x,y)\in R \Rightarrow (x,z) \in R$ für alle $x,y,z\in X$\\
\underline{symmetrisch} $\Leftrightarrow (x,y)\in R$ für alle $x,y\in X$
\subsubsection{Beispiel}
Die Relation $R_2$ aus Beispiel 1.1.4 ist reflexiv, transitiv, aber nicht Symmetrisch.  Die Relation $R_3$ aus Beispiel 1.1.5 ist reflexiv, transitiv und symmetrisch.
\subsubsection{Definition (\"{A}quivalenzrelation)}
Eine Relation $A$ auf eine Menge $X$ heißt eine \underline{\"{A}quivalenzrelation}, falls sie reflexiv, transitiv und symmetrisch ist.  F\"{u}r ein Paar $(x,y)\in A$  Schreiben wir $x\mathtt{\sim} y$ und nennen $x$ und $y$ \"{a}quivalent.
\subsubsection{Beispiel}
\begin{enumerate}
\item Sei $X$ eine beliebige Menge.  Dann ist $\{(x,y)\in X\times X:x=y\}$ eine \"{A}quivalenzrelation (\underline{Identit\"{a}tsrelation}).
\item Ebenso ist das ganze Produkt $X\times X$ eine \"{A}quivalenzrelation (\underline{Allrelation}).
\item Die Relation $R_3$ aus Beispiel 1.1.5 ist eine \"{A}quivalenzrelation.  Mit ihr lassen sich Computer nach ihrem Hersteller klassifizieren.
F\"{u}r jedes $[x]:=\{y\in X:x \mathtt{\sim} y\}$ die von $X$ erzeugte \underline{\"{A}quivalenzklasse} und ein Element $y\in [x]$ heißt \underline{Repr\"{a}sentant} von $[x]$.
\end{enumerate}
\subsubsection{Beispiel}
\begin{enumerate}
\item F\"{u}r die Identit\"{a}tsrelation ist $[x]=\{x\}$ für alle $x\in X$.  Die Allrelation besitzt genau eine \"{A}quivalenzklasse $[x]=X$.
\item Im Beispiel 1.1.5 sind die \"{A}quivalenzklassen die Menge aller Hersteller.
\end{enumerate}

\subsection{Abbildungen}
$F\subseteq D\times B$.\\
\begin{center}
\includegraphics[scale=0.4]{1-2.jpg}
\end{center}
\subsubsection{Definition (Abbildungen, Funktion)}
Eine Relation F zwischen zwei nichtleeren Mengen $D$ und $B$ heißt \underline{Abbildung} oder \underline{Funktion} von $D$ nach $B$, falls f\"{u}r alle $x\in D$ gilt.\\
1) Es existiert ein $y\in B$ mit $(x,y) \in F$\\
2) Mit $y_1,y_2\in B$ folgt aus $(x,y_1) \in F$ und $(x,y_2)\in F$, dass $y_1=y_2$.\\
Die Menge $D$ heißt \underline{Definitionsbereich}  und $B$ \underline{Bildbereich} von $F$.  Im Fall $D=B$ spricht man von einer \underline{Abbildung auf $D$} oder um einer \underline{Selbstabbildung auf $D$}.
\subsubsection{Bemerkung}
Veranschaulicht man Funktionen auf (endlichen) Mengen $D$ als gerichtete Graphen (Beispiel 1.1.6), so geht von jedem Knoten genau eine Kante ab.
Anstelle der Notation $F \subseteq D \times B,\ (x,y)\in F$ schreibt man auch $f:D\rightarrow B, x\mapsto f(x)$ oder $y:=f(x)$
Mit einer weiteren nichtleeren Menge $C$ und einer Abbildung $g:B\rightarrow C$ ist die \underline{Verknüpfung (Komposition)} von $g$ und $f$ definiert als $g\circ f: D\rightarrow C, (g\circ f)(x):=g(f(x))$.  Im Fall von Abbildungen $f,g$ auf $D$ gilt i.A. $f\circ g\not = g\circ f$.
Statt einzelner Punkte $x\in D$ kann man auch Mengen $X\subseteq D$ abbilden: $f(X):=\{y\in B:$ es gibt ein $x\in X$ mit $y=f(x)\}$.  $f(X)$ heißt \underline{Bild} von $X$ unter $f$.  Das \underline{Urbild} einer Menge $Y\subseteq B$ ist definiert durch $f^{-1}(Y):=\{x\in D\ f(x)\in Y\}$.  Eine Abbildung $f:D\rightarrow B$ heißt \\
\underline{injektiv} $\Leftrightarrow f^{-1}(\{y\}$ enthält für alle $y\in B$ höchstens ein Element \\
\underline{surjektiv} $\Leftrightarrow f^{-1}(\{y\})$ enthält für alle $y\in B$ mindestens ein Element. \\
\underline{bijektiv} $\Leftrightarrow f^{-1}(\{y\})$ enthält für alle $y\in B$ genau ein Element. \\
Eine Abbildung $f:D\rightarrow B$ ist genau dann bijektiv, wenn sie injektiv und surjektiv ist.\\
\begin{center}
\includegraphics[scale=0.4]{1-2-2.jpg}
\end{center}
\subsubsection{Beispiel (identische Abbildung)}
Die \underline{identische Abbildung} auf eine Menge $D\not = \emptyset$ ist $id_D:D\rightarrow D,id_D(x):=x$. Sie ist bijektiv.
\subsubsection*{Beispiel}
Die Relation $R_0$ aus Beispiel 1.1.2 zwischen $X=\{Land\}$ und $Y=\{Stadt\}$ ist eine Funktion $r_o :X\rightarrow Y$ $r_0(Land):=$ Hauptstadt vom Land.  Ihr Bild ist $r_0(X)=\{$Hauptst\"{a}dte$\}$ und die Urbilder lauten:
\[r_0^{-1}(\{s\}) = \begin{cases}
\emptyset & \text{falls $s$ keine Hauptstadt},\\
\{l\}& \text{falls $s$ Hauptstadt von $l$}.
\end{cases}\]
Folglich ist $r_0$ injektiv, aber nicht surjektiv.  Betrachtet man die Menge aller Haupst\"{a}dte als Bildbereich von $r_0$, so ist diese Abbildung auch surjektiv.
\subsubsection{Beispiel}
Die Relation $R_1$ zwischen $\mathbb{R}$ und $[0,\infty )$ aus Beispiel 1.1.3 ist eine Abbildung und lässt sich schreiben als $r_1:\mathbb{R}\rightarrow [0,\infty ),\ r_1(x):=\left|x\right|$
F\"{u}r sie gilt $r_1(\mathbb{R}):=[0,\infty )$ und $r_1^{-1}(\{y\})=\{-y,y\}$ f\"{u}r alle $y\in [0,\infty )$.  Also ist $r_1:\mathbb{R}\rightarrow [0,\infty )$ surjektiv, aber nicht injektiv.  Betrachten wir $r_1$ mit ganz $\mathbb{R}$ als Bildbereich, so gilt $r_1^{-1}(\{y\})=\emptyset$ f\"{u}r $y<0$ und dann ist $r_1$ nicht mehr surjektiv.
\subsubsection{Beispiel (ASCII-Code)}
Der ASCII-Code zur Codierung alpha-numerischer Zeichen ist gegeben durch eine bijektive Abbildung $f:\{0,1,\cdots ,255\text{ bzw. $127$}\}\rightarrow \{$Zeichen$\}$.\\
Einfache Beispiele (etwa Beispiel 1.2.5) zeigen, dass die Umkehrrelation $F^{-1}$ einer Abbildung $F\subseteq D\times B$ bzw. $f:D\rightarrow B$ nicht unbedingt eine Abbildung ist.
\subsubsection{Definition (Umkehrabbildung)}
Eine Abbildung $f:D\rightarrow B$ heißt \underline{umkehrbar}, falls ihre Umkehrrelation $F^{-1}$ wieder eine Abbildung ist.  F\"{u}r letztere schreibt man $f^{-1}:B\rightarrow D$ und nennt sie \underline{Umkehrabbildung} von $f$.
\subsubsection{Bemerkung}
Mit einer umkehrbaren Abbildung $f:D\rightarrow B$ ist auch ihre Umkehrfunktion $f^{-1}:B\rightarrow D$ umkehrbar mit $f^{-1}\circ f = id_D$ und $f\circ f^{-1}=id_B$.
\subsubsection{Korollar}
Eine Abbildung $f:D\rightarrow B$ ist genau dann umkehrbar, wenn $f$ bijektiv ist.  F\"{u}r umgekehrtes $f$ existiert die Umkehrfunktion nur auf $f(D)$. \\
\underline{Beweis}: Hausaufgabe
\subsection{Matrizen}
Wir führen kurz die komplexen Zahlen $\mathbb{C}$ ein.  Darunter versteht man alle Paare $z=(x,y)$ reeller Zahlen $x,y\in \mathbb{R}$ mit der \underline{Addition}: \[z_1+z_2=(x_1+x_2,y_1+y_2)\] und der \underline{Multiplikation}: \[z_1\cdot z_2:=z_1z_2=(x_1x_2-y_1y_2,x_1y_2+x_2y_1)\] wobei $z_1=(x_1,y_1),z_2=(x_2,y_2)$ \\
Differenz und Quotient ergeben sich zu:
\[ z_1-z_2=(x_1-x_2,y_1-y_2)\]
\[\dfrac{z_1}{z_2} = \left(\dfrac{x_1x_2+y_1y_2}{x_2^2+y_2^2},\dfrac{x_2y_1-x_1y_2}{x_2^2+y_2^2}\right) \text{ falls } x_2^2+y_2^2\not = 0\]
Alternative Darstellung:
\[z=(x,y) = x+iy \text{ mit der Konvention } i^2=-1\]
Wobei $x$ der Realteil ($Rez=x$) ist und $y$ der Imaginärteil ($Imz=y$).\\
\begin{center}
\includegraphics[scale=0.4]{1-3.jpg}
\end{center}
Im Folgenden stehe $\mathbb{K}$ für eine der drei Mengen $\mathbb{Q}$ (rationalen Zahlen), $\mathbb{R}$ (reelle Zahlen) oder $\mathbb{C}$.
\subsubsection{Definition (Matrix)}
Eine $m\times m$-Matrize ist ein rechteckiges Schema von Zahlen $a_{ij}\in \mathbb{K}$ der Form \[A=(a_{i,j}) 1\leq i\leq m,1\leq j\leq m =\left( \begin{array}{cccc}a_{1,1}& a_{1,2}& \cdots & a_{1,m}\\ a_{2,1}& a_{2,2}& \cdots & a_{2,m} \\ \vdots & \vdots & \ddots & \vdots \\ a_{m,1}& a_{m,2}& \cdots & a_{m,m}\end{array}\right) \]
Der erste Index $i\in\{1,\cdots ,m\}$ nummeriert die $m$ \underline{Zeilen}, der zweite Index $j\in \{1,\cdots ,m\}$ die $m$ Spalten der Matrix $A$, das \underline{Element} $a_{ij}\in \mathbb{K}$ steht daher in der $i$-ten Zeile und der $j$-ten Spalte.  F\"{u}r die Menge aller solchen Matrizen schreiben wir $\mathbb{K}^{m\times m}$. Für eine \underline{quadratische Matrix} A gilt $m = n$ und die $a_{i,i}$ heißen \underline{Diagonalelement}.
\[A'=\left( \begin{array}{ccc}a_{1,1}& \cdots & a_{1,n}\\\vdots & \ddots & \vdots\\a_{m,1}& \cdots & a_{m,n}\end{array}\right)\]
\subsubsection{Beispiel ($n$ Tupeln, $m$-Spalten)}
Ein \underline{$n$-Tupel} $x=(x_1,\cdots ,x_n)$ von Zahlen $X$ aus $\mathbb{K}$ wird als $1\times m$-Matrix interpretiert.  Eine \underline{$m$-Spalte} $x=\left(\begin{array}{c}x_1\\\vdots \\x_m\end{array}\right)$
wird als $m\times 1$-Matrix verstanden, identifiziert durch $\mathbb{K}^m=k^{m\times 1}$.
\subsubsection{Kronecker-Symbol, Einheits- und Nullmatrixe)}
Wir definieren das \underline{Kronecker-Symbol} $S_{i,j}:=\begin{cases}1,i=j\\0,i\not= j\end{cases}$ und $I_m:=(S_{i,j})_{1\leq i,j\leq m}$ ist die \underline{Einheitsmatrix}.  Bei der Nullmatrix $0=(0)_{\substack{1\leq i\leq m\\1\leq j\leq n}}$ sind alle Elemente gleich $0\in \mathbb{N}$.
\subsubsection{Beispiel (Diagonal- und Dreieckmatrizen)}
Man nennt eine quadratische Matrix $A=(a_{i,j})_{1\leq i,j\leq n}$ \underline{diagonal} falls $a_{i,j}=0$ für $i\not=j$.  Wir schreiben dann $A=\left(\begin{array}{cccc}a_{1,1} & 0 & \cdot & 0\\ 0 & a_{2,2} &\cdot & 0\\ 0 &\cdot &\cdot & a_{n,n}\end{array}\right)=\text{diag}(a_{1,1},\cdot,a_{n,n})$.  Eine \underline{obere Dreiecksmatrix} ist quadratisch und erfüllt $a_{i,j}=0$ für $i>j$, wogegen eine \underline{untere Dreiecksmatrix} $a_{i,j}=0$ für $i<j$ erfüllt.  Sie sind von der Form: $A=\left(\begin{array}{cccc}a_{1,1} & a_{1,2} & \cdots & a_{1,n}\\ 0 & a_{2,2} & \cdots & a_{2,n}\\ \vdots & \vdots & \ddots &\vdots \\ 0 & \cdots & 0 & a_{n,n}\end{array}\right)$ bzw. $A=\left(\begin{array}{cccc}a_{1,1} & 0 & \cdots & 0\\a_{2,1} & a_{2,2} & \cdots & 0 \\ \vdots & \vdots & \ddots &\vdots \\ a_{n,1} & a_{n,2}  &\cdots & a_{n,n}\end{array}\right)$.\\
Mathematische Operationen für Matrizen:
\begin{itemize}
\item \underline{Skalare Multiplikation}: $\mathbb{K}\times \mathbb{K}^{m\times n} \rightarrow \mathbb{K}^{m\times n}, \alpha \cdot A= \alpha A=(\alpha a_{i,j})_{\substack{1\leq i\leq m \\ i\leq j\leq n}}$.  Wir schreiben $-A:=(-1)\cdot A$
\item \underline{Addition}: $+:\mathbb{K}^{m\times n}\times \mathbb{K}^{m\times n} \rightarrow \mathbb{K}^{m\times n},A+B=(a_{i,j}+b_{i,j})_{\substack{1\leq i\leq m \\ 1\leq j\leq n}}$.  Die Subtraktion lautet $A-B=A+(-B)$.
\item Genau für $m\times n$-Matrizen $A$ und $n\times p$-Matrizen $B$ lässt sich eine \underline{Multiplikation} erklären.  $\cdot : \mathbb{K}^{m\times n}\times \mathbb{K}^{n\times p}\rightarrow \mathbb{K}^{m\times p}$.  $A\cdot B=AB:=(\sum^n_{k=1} a_{i,k}b_{k,j})_{\substack{1\leq i\leq m\\1\leq j\leq p}}$. das Produkt ist also eine $m\times p$-Matrix.
\end{itemize}
\underline{Merke}: Das Produkt macht nur Sinn, falls die Spaltenzahl der ersten mit der Zeilenzahl der zweiten Matrix übereinstimmt.
\subsubsection{Bemerkung}
(1) Um Produkte von Matrizen $A\in \mathbb{K}^{m\times n}$ und $B\in \mathbb{K}^{n\times p}$ zum berechnen ergibt sich das Schema $\begin{array}{ccc} & | & B\\A & | & C\end{array} C=(\sum^{n}_{k=1} a_{i,k}b_{k,j})_{\substack{1\leq i\leq m\\1\leq j\leq p}}$\\
(2) Spezialfall: $A\in \mathbb{K}^{m\times m},x\in \mathbb{K}^m$ $Ax=\sum^{m}_{k=1}\left(\begin{array}{cc}a_{1,k}&x_{1}\\\vdots & \vdots \\ a_{m,k} & x_k\end{array}\right)$.
\subsubsection{Beispiel}
Das Produkt von $A=\left(\begin{array}{cc}0 & 1\\ 2 & 3\end{array}\right)$ und $B=\left(\begin{array}{cc}4 & 5 \\ 6 & 7\end{array}\right)$ lautet:
\[\begin{array}{c|cc} & \begin{matrix}4\\ 6\end{matrix} & \begin{matrix} 5\\ 7\end{matrix} \\ \hline \\0\ 1 & 0\cdot 4+1\cdot 6 & 5\cdot 0 + 1\cdot 7 \\ 2\ 3 & 2\cdot 4+6\cdot 3 & 2\cdot 5+7\cdot 3\end{array}\] also $C=AB=\left(\begin{array}{cc}6 & 7 \\ 26 & 31\end{array}\right)$.\\
Im Umgekehrter Reihenfolge gilt $BA=\left(\begin{array}{cc} 10 & 19 \\ 14 & 27\end{array}\right)$.  Daher ist das Produkt von Matrizen nicht kommutativ $AB\not= BA$.
\subsubsection{Beispiel}
(1) Für $A\in \mathbb{K}^{m\times n}$ gilt $I_m A=A=AI_m$\\
(2) Für $A=\left(\begin{array}{cc}0 & 1\\ 0 & 0\end{array}\right)$ und $B=\left(\begin{array}{cc}1 & 0\\ 0 & 0\end{array}\right)$ gilt $AB=0$, womit das Produkt von Matrizen nicht \underline{nullteilerfrei} ist, d.h. $AB=0$ kann gelten, ohne dass ein Faktor Null ist.\\
(3) Das Produkt von $\left(\begin{array}{ccc}0 & 1 & 2 \\ 3 & 4 & 5\end{array}\right)$ und $\left(\begin{array}{cc}6 & 7 \\ 8 & 9\end{array}\right)$ ist nicht definiert,
$\left(\begin{array}{cc}6 & 7 \\ 8 & 9\end{array}\right)\left(\begin{array}{ccc}0 & 1 & 2 \\ 3 & 4 & 5\end{array}\right) = \left(\begin{array}{ccc}21 & 34 & 47 \\ 27 & 44 & 61\end{array}\right)$ dagegen schon.
\subsubsection{Beispiel (RGB - Raum)}
Im RGB-Farbmodell werden Farben durch Tupel $(r,g,b)$ reeller Zahlen $r,g,b\in \mathbb{R}$ beschreiben: $(1,0,0)$ = rot, $(0,0,1)$ blau, $(1,1,0)$ gelb.  Alternativ: $YIQ$-Modell $(y,i,q)$. \\
Umrechnung $\left(\begin{array}{c}y\\i\\q\end{array}\right)=\left(\begin{array}{ccc}0.3 & 0.6 & 0.1\\ 0.6 & -0.3 & -0.3\\ 0.2 & -0.5 & 0.3\end{array}\right) \left(\begin{array}{c}r\\g\\b\end{array}\right)$.
\subsubsection{Beispiel (Inzedenzmatrix)}
Gerichtete Graphen ohne Schleifen (kein Knoten wird durch eine Kante mit sich selbst verbunden, siehe Bemerkung 1.1.6) mit den Knoten ${\hat{1},\cdots, \hat{m}}$ mit den Knoten ${1,\cdots ,m}$ lassen sich durch eine sogenannte \underline{Inzedenzmatrix} $A\in \mathbb{K}^{m\times n}$ beschreiben mit
\[a_{i,j}=\begin{cases}1,\text{ Von Knoten }\hat{1}\text{ geht die Kante }j\text{ aus.}\\ -1,\text{ ein Knoten }\hat{1}\text{ mündet die Kante }j\\0,\text{ Knoten }\hat{1}\text{ und Kante }j\text{ berühren sich nicht.}\end{cases}\]. 
\begin{tikzpicture}[node distance=2cm, auto]
\node (1) {$\hat{1}$};
\node (2) [right of = 1] {$\hat{2}$};
\node (3) [below of = 2] {$\hat{3}$};
\draw[decoration={markings,mark=at position 1 with {\arrow[ultra thick]{>}}},postaction={decorate}] (1) to node {1} (2);
\draw[decoration={markings,mark=at position 1 with {\arrow[ultra thick]{>}}},postaction={decorate}] (2) to node {3} (3);
\draw[decoration={markings,mark=at position 1 with {\arrow[ultra thick]{>}}},postaction={decorate}] (3) to node {2} (1);
\end{tikzpicture}
$A=\left(\begin{array}{ccc}1& -1 & 0\\ -1 & 0 & 1\\ 0 & 1 & -1\end{array}\right)$ \\
\begin{tikzpicture}[node distance=2cm, auto]
\node (1) {$\hat{1}$};
\node (2) [right of = 1] {$\hat{2}$};
\node (3) [below of = 2] {$\hat{3}$};
\draw[decoration={markings,mark=at position 1 with {\arrow[ultra thick]{>}}},postaction={decorate}] (1) to node {1} (2);
\draw[decoration={markings,mark=at position 1 with {\arrow[ultra thick]{>}}},postaction={decorate}] (1) to node {2} (3);
\end{tikzpicture} $A=\left(\begin{array}{cc}1 & 1\\ -1 & 0 \\ 0 & -1\end{array}\right)$ \\
\begin{tikzpicture}[node distance=2cm, auto]
\node (1) {$\hat{1}$};
\node (2) [right of = 1] {$\hat{2}$};
\node (3) [below of = 1, left of = 2] {$\hat{3}$};
\draw[decoration={markings,mark=at position 1 with {\arrow[ultra thick]{>}}},postaction={decorate}] (1) to node {1} (2);
\draw[decoration={markings,mark=at position 1 with {\arrow[ultra thick]{>}}},postaction={decorate}] (1.290) to node {2} (3.70);
\draw[decoration={markings,mark=at position 1 with {\arrow[ultra thick]{>}}},postaction={decorate}] (3.90) to node {3} (1.270);
\draw[decoration={markings,mark=at position 1 with {\arrow[ultra thick]{>}}},postaction={decorate},loop below] (3) to node {4} (3);
\end{tikzpicture} Nicht schleifen frei!
\subsubsection{Satz (Rechenregeln für Matrizen)}
Für Zahlen $\alpha \in \mathbb{K}$ und Matrizen $A\in \mathbb{K}^{m\times n},B\in\mathbb{K}^{m\times p}$ gilt das \underline{Distributiv-Gesetz}.  $A(B+C)=AB+AC$ für alle $C\in \mathbb{K}^{m\times p}$ und die \underline{Assoziativ-Gesetze} $(\alpha A)B=A(\alpha B), A(BC)=(AB)C$ für alle $C\in \mathbb{K}^{p\times q}$.
Beweis: Übung.
\subsection{Lineare Gleichungen}
\subsubsection{Definition (lineare Gleichung)}
Es seien $A\in \mathbb{K}^{m\times n}$ und $b\in \mathbb{K}^{m}$.  Dann bezeichnet man $(L_b)\ Ax=b$ als lineares \underline{Gleichungssystem} mit $m$ Gleichungen für die $n$ unbekannten $x_m\in \mathbb{K}$ oder kurz also \underline{lineare Gleichung} in $\mathbb{K}^m$.  $A$ heißt \underline{Koeffizientenmatrix} und $b$ \underline{Inhomogenität} von $(L_b)$.  Im Fall $b\not= 0$ nennt man $(L_b)$ \underline{inhomogen} und erhält andernfalls die \underline{homogene Gleichung}: $(L_0)\ Ax=0$.  Eine Lösung von $(L_b)$ ist ein Element $x\in \mathbb{K}^m$ mit $Ax=b$ und $L_b:=\{ x\in \mathbb{K}^m:Ax=b\}$ steht für die \underline{Lösungsmenge} von $(L_b)$.
\subsubsection{Bemerkung}
(1) Ausgeschrieben lautet $(L_b)$:\\$a_{1,1}x_1+a_{1,2}x_2+\cdots +a_{1,n}x_n = b_1\\a{2,1}x_1+a_{2,2}x_2+\cdots +a_{2,n}x_n = b_2\\\cdots \\a{m,1}x_1+a_{m,2}x_2+\cdots +a_{m,n}x_n=b_m$.\\Oder noch unübersichtlicher $\sum^n_{j=1} a_{i,j}x_j = b_i$ für $1\leq i\leq m$.\\
(2) $(L_b)$ hat stehts die \underline{triviale Lösung} $0\in \mathbb{K}^m$.  Inhomogene Gleichungen müssen nicht unbedingt lösbar sein: $0x=1$.\\
\subsubsection{Satz (Superpositionsprinzip)}
Es seien $x,y \in \mathbb{K}^n$ Lösungen von ($L_0$).  Dann ist auch $\alpha x+\beta y$ eine Lösung von ($L_0$), d.h. $\alpha x+\beta y\in L_0$ für alle $\alpha ,\beta \in \mathbb{K}$. \\
Beweis: Übung.
\subsubsection{Satz}
Ist $\hat{x}\in\mathbb{K}^n$ eine Lösung von ($L_b$) so gilt $L_b=\hat{x}+L_0$.  Hierbei: Für gegebene $x\in\mathbb{K}^n,A\subseteq \mathbb{K}^n$ ist $x+A:=\{y\in \mathbb{K}^n:\text{ es gibt ein }a\in A\text{ mit } y=x+a\}$\\
Beweis: Übung.
Nun: Explizite Lösung von ($L_b$)!\\
Besonders einfach, falls $A\in\mathbb{K}^{m\times n}$ diagonal ist gilt nämlich $a_{i,i}\not=0,1\leq i\leq n$, so besitzt ($L_b$) die eindeutige Lösung $x\in\mathbb{K}^n$ mit Elementen $X_1=\dfrac{b_i}{a_{i,i}}$ für $1\leq 1\leq n$ ist dagegen $d_{i,i}=0$ für ein $1\leq i\leq n$, so besitzt ($L_b$) unendlich viele Lösungen für $b_i=0$ und anderenfalls keine Lösung.\\
\underline{Allgemeinere Klasse}: Ein $A\in \mathbb{K}^{m\times n}$ ist in \underline{Zeilen-Stufen-Form} (ZSF) falls in jeder Zeile gilt:
\begin{enumerate}
\item[(1)] Beginnt sie mit $k$ Nullen, so stehen unter diesen Nullen lediglich weitere Nullen.\\
\item[(2)] Unter dem ersten Element $\not= 0$ stehen nur Nullen.\\
\end{enumerate}
Bei \underline{strenger ZSF } muss zusätzlich gelten:
\begin{enumerate}
\item[(3)] Über jedem ersten Element $\not= 0$ stehen nur Nullen\\
\end{enumerate}
\subsubsection{Beispiel}
\begin{enumerate}
\item[(1)] Obere Dreiecksmatrizen sind in ZSF, Diagonalmatrizen sogar in strenger ZSF.\\
\item[(2)] Bezeichnet $*$ ein Element $\not= 0$, so gilt:
\begin{itemize}
\item $\left(\begin{array}{ccc}* & * & *\\ * & * & *\\ 0 & 0 & *\end{array}\right)$,
$\left(\begin{array}{ccc}0 & * & *\\ 0 & * & *\\ 0 & * & *\end{array}\right)$,
$\left(\begin{array}{ccc}* & 0 & 0\\ * & * & 0\\ * & * & *\end{array}\right)$
sind nicht in ZSF.
\item $\left(\begin{array}{cccc}* & * & * & *\\ 0 & * & * & *\\ 0 & 0 & 0 & *\end{array}\right)$ ist in ZSF (aber nicht strenger ZSF).
\item $\left(\begin{array}{cccc}* & * & 0 & 0\\ 0 & 0 & * & 0\\ 0 & 0 & 0 & 0\end{array}\right)$ ist in strenger ZSF
\end{itemize}
\end{enumerate}
\subsubsection{Beispiel (Rückwärts-Substitution)}
Die inhomogene lineare Gleichung $(1.4b)\begin{cases}x_1+2x_2+3x_3+4x_4=1,\\ x_2+2x_3+3x_4=1,\\x_3+2x_4=1\end{cases}$ hat die Koeffizientenmatrix bzw. Inhomogenität $A=\left(\begin{array}{cccc}1 & 2 & 3 & 4 \\ 0 & 1 & 2 & 3\\ 0 & 0 & 1 & 2\end{array}\right), b=\left(\begin{array}{c}1\\ 1\\ 1\\\end{array}\right)$\\
\underline{Rückwärtssubstitution}: Aus der letzten Gleichung $x_3+2x_4=1$ sieht man, dass $x_4=t$ frei gewählt wenden kann, $t\in \mathbb{K}$.  Dies liefert $x_3=1-2t$.  Die bekannten variablen $x_3,x_4$ können in die zweite Gleichung von ($1.4b$) eingesetzt werden, also $x_2 = 1-2x_3-3x_4=t-1$ und analog liefert die erste Gleichung $x_1=1-2x_2-3x_3-4x_4=0$.  Die Lösungsmenge von ($1.4b$) ist also:\[L_b=\left\{\left(\begin{array}{c}0\\ t-1\\ 1-2t\\ t\end{array}\right)\in \mathbb{K}^4:t\in\mathbb{K}\right\} = \left(\begin{array}{c}0\\ -1\\ 1\\ 0\end{array}\right)+\mathbb{K}\left(\begin{array}{c}0\\ 1\\ -2 \\1\end{array}\right)\]
Die Lösungsmenge $L_b$ von ($L_b$) ändert sich nicht, wenn folgende Operationen auf ($1.4b$) angewandt werden:
\begin{itemize}
\item Vertauschen von Gleichungen
\item Multiplikation von Gleichungen mit $\alpha \in \mathbb{K}\\\{0\}$
\item Addition des $\alpha$-fachen der $k$-ten Gleichung zur $j$-ten.\\
Diese sind \underline{elementare Zeilentransformationen}.
\end{itemize}
ZIEL: Transformiere $A$ bzw. ($L_b$) auf ZSF mittels elementarer Zeilentransformationen.  Systematisch: Gauß Algorithmus.\\
Zu seiner Beschreibung gehen wir davon aus, dass die erste Spalte von $A$ von $0$ verschieden ist (anderenfalls sind $x_1,\cdots ,x_n$ umzunummerieren).  Ohne Sonderfälle zu berücksichtigen gilt:
\begin{enumerate}
\item Ordne die Gleichungen in ($1.4a$) so an, dass $a_m\not= 0$.  In der gängigen Notation schreibt man nun ($1.4b$) als $\begin{array}{cccc|c}a_{1,1} & a_{1,2} & \cdots & a_{1,n} & b_1\\a_{2,1} & a_{2,2} & \cdots & a_{2,n} & b_2\\\vdots & \vdots & \ddots &\vdots &\vdots \\a_{m,1} & a_{m,2} & \cdots & a_{m,n} & b_m\end{array}$
\item Subtrahiere von der $i$-ten Gleichung, $2\leq i\leq m$ in ($1.4a$) das $\dfrac{a_{i,1}}{a_{1,1}}$-fache der ersten Gleichung: 
\[\begin{array}{cccc|c}a_{1,1} & a_{1,2} & \cdots & a_{1,n} & b_1\\0 & a_{2,2}^{(1)} & \cdots & a_{2,n}^{(1)} & b_2^{(1)}\\ \vdots &\vdots &\ddots &\vdots &\vdots \\0 & a_{m,2}^{(2)} & \cdots & a_{m,n}^{(1)} & b_m^{(1)}\end{array} \begin{cases}a_{1,1}x_1+\cdots + a_{1,n}x_n=b_1 \\ A^{(1)}x^{(1)} = b^{(1)}\end{cases}\]
mit $A^{(1)}\in\mathbb{K}^{(m-1)\times (n-1)},b\in \mathbb{K}^{m-1}$.
\item Transformiere $A^{(1)}x^{(1)} = b^{(1)}$ entsprechend und fahre sukzessive fort, bis (idealerweise) eine Dreiecks- oder ZSF entstanden ist.
\item Löse das resultierende System durch Rückwärts-Substitution.
\end{enumerate}
\subsubsection{Beispiel}
Als Kurzschreibweise für \[(1.4d)\begin{cases}x_1+2x_2+3x_3=0\\ 4x_1+5x_2+6x_3=0\\ 7x_1+8x_2+9x_3=0\end{cases}\]
\[\begin{array}{ccc|c}1 & 2 & 3 & 0\\ 4 & 5 & 6 & 0\\ 7 & 8 & 9 & 0\end{array} \begin{array}{c} \\ II - 4I\\ II-7I\end{array} \Leftrightarrow \begin{array}{ccc|c}1 & 2 & 3 & 0\\ 0 & -3 & -6 & 0\\ 0 & -6 & -12 & 0\end{array} \begin{array}{c} \\ (-\frac{1}{3})\\ III-2II\end{array} \Leftrightarrow \begin{array}{ccc|c}1 & 2 & 3 & 0\\ 0 & 1 & 2 & 0\\ 0 & 0 & 0 & 0\end{array}\]
Damit ist ($1.4d$) äquivalent zu $\begin{cases}x_1+2x_2+3x_3=0\\ x_2+2x_3=0\end{cases}$\\
Rückwärts-Substitution: Wähle $x_3=t$ mit $t\in\mathbb{K}$ und es folgt $x_2=-2x_3=-2t,x_1=-2x_2+3x_3=t$.  Die Lösungsmenge von ($1.4d$) ergibt sich zu:
\[ L_0=\left\{\left(\begin{array}{c}1\\ -2\\ 1\end{array}\right) \in \mathbb{K}^3:t\in\mathbb{K}\right\} = \mathbb{K}\left(\begin{array}{c}1\\ -2\\ 1\end{array}\right)\]
\subsubsection{Satz}
Hat ($L_0$) weniger Gleichungen als Unbekannte, also $m<n$, so besitzt sie unendlich viele Lösungen.
\underline{Beweis}:
\renewcommand{\labelenumi}{\Roman{enumi}.}
\begin{enumerate}
\item Man zeigt (*) ($L_0$) hat eine nichttriviale Lösung.
\item Da ($L_0$) nach Schritt (I) eine Lösung $x\not= 0$ besitzt ist nach dem superpositionsprinzip aus Satz 1.4.3 auch jeder $tx,t\in \mathbb{K}$, eine Lösung \#.
\end{enumerate}
\subsubsection{Satz}
Besitzt ($L_b$) genauso viele Gleichungen wie Unbekannte, also $m=n$, so gilt:
\renewcommand{\labelenumi}{(\alph{enumi})}
\begin{enumerate}
\item Ist $L_0=\{0\}$, so besitzt ($L_b$) genau eine Lösung.
\item Besitzt ($L_0$) eine nichttriviale Lösung, so existieren entweder keine oder unendlich viele verschiedene Lösungen von ($L_b$)
\end{enumerate}
\underline{Beweis}:
\begin{enumerate}
\item Wie gehen mittels vollständiger Induktion vor.\\
Für $n=1$ gilt die Behauptung offenbar.  Im Induktionsschritt gelte (a) für $n-1$.  Da ($L_0$) nur die triviale Lösung hat gilt $A\not= 0$.  Durch Umnummerieren erreichen wir $a_{1,1}\not= 0$.  Dann wird zur $i$-ten Gleichung, $2\leq i$, in ($1.4a$) das $-\frac{a_{i,1}}{a_{1,1}}$-fache der ersten Gleichung addiert:
\[(1.4f)\begin{cases}a_{1,1}x_1+\cdots +a_{1,n}x_n = b_1\\ A^*=\left[\begin{array}{c}x_2\\ \cdots \\ x_n\end{array}\right]= b^*\end{cases} \text{ mit }A^*\in \mathbb{K}^{(n-1)\times (m-1)}, b^*\in \mathbb{K}^{n-1} \]
Beweis: \\
Wir wissen:
\begin{enumerate}
\item Die homogene Gleichung $A^*x^* = 0$ hat nur die triviale Lösung, denn sonst hätte ($L_0$) eine nicht triviale Lösung.  Das Teilsystem $A^*x^* = b^*$ besitzt nach Induktionsannahme genau eine Lösung $x^*$ mit Elementen $x_2,\cdots ,x_m$.  Durch Einsetzten in die erste Gleichung in ($1.4f$) folgt ein eindeutiger Wert $x_1$ und die Lösung von ($L_b$) in eindeutiger Weise.
\item Es sei $\hat{x}$ eine Lösung von ($L_b$) und $x$ eine nichttriviale Lösung von ($L_o$).  Dann liefern die Sätze $1.4.3$ und $1.4.4$, dass $\hat{x}+\alpha x$ die Gleichung löst für jedes $\alpha \in \mathbb{K}$.  In diesem Fall hat ($L_b$) unendlich viele Lösungen.  Die einzige verbleibende Möglichkeit ist, dass ($L_b$) keine Lösung besitzt.
\end{enumerate}
\end{enumerate}
