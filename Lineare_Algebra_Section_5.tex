\section{Innere Produkte}
In diesem Kapitel sei stets $\mathbb{K}=\mathbb{R}$ oder $\mathbb{K}=\mathbb{C}$.  Wir beschränken uns weiter auf reelle oder komplexe lineare Räume $X$.  Insbesondere im Fall $\K = \Co$ erinnern wir an das komplex-konjugierte
\[\overline{z}=(x,-y)=x-iy\]
einer komplexen Zahl $z=(x,y)=x+iy$, sowie ihren Betrag $\left|z\right|:=\sqrt{zz}=\sqrt{x^2+y^2}$.  Wir betrachten $\R$ als Teilmenge von $\Co$ und erhalten: $z=\overline{z}\Leftrightarrow z\in \R$.
\subsection{Skalarprodukte und Orthogonalität}
\subsubsection{Definition (inneres Produkt)}
Ein \underline{inneres Produkt} auf $X$ ist eine Abbildung $\langle\cdot ,\cdot \rangle\colon X\times X\rightarrow \K$ mit
\romannum
\begin{enumerate}
\item $\langle\alpha x+\beta y,z\rangle=\alpha \langle x,z\rangle+\beta \langle y,z\rangle$ (\underline{Linearität im 1. Argument})
\item $\langle x,y\rangle=\overline{\langle y,x\rangle}$ (\underline{konjugierte Symmetrie})
\item $\langle x,x\rangle\geq 0$ und Gleichheit genau für $x=0$ (\underline{positive Definitheit})
\end{enumerate}
Für alle $x,y,z\in X,\ \alpha,\beta \in \K$.  Ein linearer Raum mit innerem Produkt heißt auch \underline{Prä-Hilbert-Raum}.\\
Statt innerem Produkt sagt man auch \underline{Skalarprodukt}.
\subsubsection{Bemerkung}
\numbers
\begin{enumerate}
\item Aufgrund der konjugierten Symmetrie ist stets $\langle x,x\rangle\in \R$, während die positive Definitheit $\langle x,x\rangle>0$ für $x\not=0$ garantiert.
\item Ein inneres Produkt ist \underline{Semilinear} im 2. Argument:
\[\phantomsection\label{5.1a}(5.1a)\ \langle x,\alpha y+\beta z\rangle=\overline{\alpha }\langle x,y\rangle+\overline{\beta }\langle x,z\rangle\text{ für alle } x,y,z \in X,\ \alpha, \beta \in \K\]
\item Unter einer \underline{Norm} auf $X$ versteht man die Funktion
\[\|\cdot \|\colon X\rightarrow \R,\ \|x\| :=\sqrt{\langle x,x\rangle}\]
Insbesondere gilt $\|x\|=0\Leftrightarrow x=0$.  Zu jedem $x\not=0$ nennen wir $y:=\frac{1}{\|x\|}x$ den \underline{normierten Vektor} zu $x$, denn $\|y\|=1$.
\end{enumerate}
\subsubsection{Bemerkung (Orthogonalität)}
\begin{enumerate}
\item Die Elemente $x,y\in X$ heißen \underline{orthogonal}, falls $\langle x,y\rangle=0$.  Die resultierende Relation 
\[x \bot y :\Leftrightarrow \langle x,y\rangle=0\]
ist symmetrisch aber nicht transitiv.  Wegen $\langle x,0\rangle=\langle0,x\rangle=0$ ist $0\in X$ orthogonal zu jedem $x\in X$.
\item Die Teilmengen $Y_1,Y_2\subseteq X$ heißen \underline{orthogonal}, insofern
\[\langle y_1,y_2\rangle=0\text{ für alle }y_1\in Y_1,\ y_2\in Y_2\]
\end{enumerate}
\subsubsection{Beispiel (Euklidischer Raum)}
$\R ^n$ mit $\langle x,y\rangle:=\sum _{j=1}^n x_jy_,$, $\|x\|=\sqrt{\sum _{j=1}^nx_j^2}$
\subsubsection{Beispiel (Unitärer Raum)}
$\Co ^n$ mit $\langle x,y\rangle=\sum _{j=1}^n x_j \overline{y_j}$, $n=1$, $\langle i,i\rangle=i\cdot (-1)=1$.