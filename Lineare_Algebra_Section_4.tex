\section{Eigenwerte}
Im gesamten Kapitel sei $X$ ein linearer Raum über dem Körper $\mathbb{K}$.
\subsection{Determinanten}
Wir beginnen mit einem Exkurs über Permutationen.  Dazu sei $(S_n,\circ )$ die in \hyperref[symmetrische]{Beispiel \ref*{symmetrische}} eingeführte symmetrische Gruppe aller bijektiven Selbstabbildungen von $\{1,\cdots ,n\},\ n\in\mathbb{N}$.  Ihre Elemente $\sigma$ werden als \underline{Permutationen} bezeichnet (von $\{1,\cdots ,n\}$) und man notiert sie als Schema:
\[\begin{bmatrix}1 & 2 & \cdots & n\\ \sigma (1) & \sigma (2) & \cdots & \sigma (n)\end{bmatrix}\]
oder als $n-$Tupel ($\sigma (1),\cdots ,\sigma (n)$).  Weil $\sigma$ bijektiv ist, kommt jede Zahl $j\in\{1,\cdots ,n\}$ genau einmal in $(\sigma (1),\cdots ,\sigma (n))$ vor; ferner gibt es genau $n!$ solche Permutationen.
\subsubsection{Definition (Signum)}
Es sei $\sigma \in S_n$ und $s(\sigma )$ bezeichnet die Anzahle der Paare $(i,j)\in\mathbb{N}^2$ mit $1\leq i<j\leq n$ und $\sigma (i) > \sigma (j)$.  Dann ist das \underline{Signum} einer Permutation $\sigma$ definiert durch
\[\mathrm{sgn}\sigma := (-1)^{s(\sigma )}\]
\subsubsection{Beispiel}
\label{4.1.2}
Für die identische Permutation id$=(1,2,\cdots ,n)$ gilt $s(\sigma )=0$ und folglich sgnid$=1$.  Weiter erhält man 
\begin{align*}
\sigma &=(2,1,3,4,\cdots ,n),\ s(\sigma ),\ \mathrm{sgn}\sigma =-1\\
\sigma &=(n,n-1,\cdots ,2, 1),\ s(\sigma )=\frac{(n-1)n}{2},\mathrm{sgn}\sigma =(-1)^{s(\sigma )}
\end{align*}
\subsubsection{Proposition}
Für alle $\sigma ,\tau \in S_n$ gilt sgn$\sigma \circ \tau =$ sgn$\sigma\cdot$sgn$\tau$.
\subsubsection{Bemerkung}
Mittels \hyperref[4.1.2]{Beispiel \ref*{4.1.2}} ist $1=$sgnid$=$sgn$\sigma \circ \sigma ^{-1}$ und damit erhalten wir
\[\label{4.1a} (4.1.a)\ \mathrm{sgn}\sigma ^{-1}=\mathrm{sgn}\sigma\text{ für alle }\sigma \in S_n\]
\underline{Beweis}: Es seien $\sigma ,\tau \in S_n$ und $x_1,\cdots ,x_n\in\mathbb{Q}$ paarweise verschieden.  Dann sind auch $y_i:=x_{\sigma (i)}$ mit $1\leq i\leq n$ paarweise verschieden.
\renewcommand{\labelenumi}{(\Roman{enumi})}
\begin{enumerate}
\item Zunächst gilt die Identität
\[\label{4.1b} (4.1b)\ \mathrm{sgn}\sigma =\prod _{1 \leq i<j \leq n} \frac{x_{\sigma (i)}-x_{\sigma (j)}}{x_i -x_j}\]
denn Zähler und Nenner des Produkts stimmen bis auf ihr Vorzeichen überein.  Im Zähler tritt ein Faktor $x_k -x_l$ mit $k>l$ genau $s(\sigma )$-mal auf, während dies im Nenner nicht vorkommt.
\item Aufgrund von \hyperref[4.1b]{$4.1b$} erhalten wir
\[\mathrm{sgn}\tau =\prod _{1\leq i<j\leq n} \frac{y_{\tau (i)} -y_{\tau (j)}}{y_i - y_j}=\prod _{1\leq i<j\leq n} \frac{x_{\sigma \circ \tau (i)}-x_{\sigma \circ \tau (j)}}{x_{\sigma (i)}-x_{\sigma (j)}}\]
\end{enumerate}
und folglich resultiert die Behauptung aus
\begin{align*}
\mathrm{sgn} \sigma\circ\tau &\stackrel{(4.1b)}{=} \prod _{1\leq i<j\leq n}\frac{x_{\sigma \circ \tau (i)}-x_{\sigma\circ\tau (j)}}{x_i -x_j}\\
&= \left(\prod _{1\leq i<j\leq n} \frac{x_{\sigma\circ\tau (i)}-x_{\sigma\circ\tau (j)}}{x_{\sigma (i)}-x_{\sigma (j)}}\right) \left(\frac{x_{\sigma (i)}-x_{\sigma (j)}}{x_i-x_j}\right)\\
&= \mathrm{sgn}\tau \cdot \mathrm{sgn}\sigma
\end{align*}
\subsubsection{Definition (Determinante)}
Die durch det:$\mathbb{K}^{n\times n}\rightarrow \mathbb{K}$
\[\label{4.1c}(4.1c)\ \mathrm{det}A:=\sum _{\sigma\in S_n} \mathrm{sgn}\sigma \prod _{i=1}^na_{i\sigma (i)}\]
\subsubsection{Bemerkung}
\renewcommand{\labelenumi}{(\arabic{enumi})}
\begin{enumerate}
\item \hyperref[4.1c]{$(4.1c)$} heißt auch \underline{Leibniz-Formel}
\item Es gilt die Beziehung $\alpha ^n$det$A$ für alle $\alpha\in\mathbb{K},A\in\mathbb{K}^{n\times n}$, folglich ist die Determinante für $n\geq 2$ nicht linear.
\end{enumerate}
\subsubsection{Beispiel}
Wir erhalten det$0=0$ und det$I_n=1$
\subsubsection{Beispiel}
In Dimensionen $n\leq 3$ kann die Determinante einer Matrix $A\in\mathbb{K}^{n\times n}$ verhältnismäßig einfach berechnet werden.
\begin{enumerate}
\item Für $n=1$ gilt det$A=a_{1,1}$
\item Für $n=2$ ist $S_2=\{(1,2),(2,1)\}$ und wir erhalten det$A=$det$\begin{pmatrix}a_{1,1} & a_{1,2}\\ a_{2,1} & a_{2,2}\end{pmatrix}=a_{1,1}a_{2,2}-a_{2,1}a_{1,2}$
\item Für $n=3$ gilt $S_3=\{(1,2,3),(2,3,1),(3,1,2),(2,1,3),(3,2,1),(1,3,2)\}$ wobei die ersten drei Permutationen das Signum $1$ besitzen und die weiteren das Signum $-1$ besitzen.  Dies liefert die  \underline{Regel von Sarrus}
\begin{align*}
\mathrm{det}A&=\mathrm{det}\begin{pmatrix}a_{1,1} & a_{1,2} & a_{1,3}\\ a_{2,1} & a_{2,2} & a_{2,3}\\ a_{3,1} & a_{3,2} & a_{3,3}\end{pmatrix} \left|\begin{matrix}a_{1,1} & a_{1,2}\\ a_{2,1} & a_{2,2}\\ a_{3,1} & a_{3,2}\end{matrix} \leftarrow \text{Bildlich dargestellt wie die Regel funktioniert}\right|\\
&=a_{1,1}a_{2,2}a_{3,3}+a_{1,2}a_{2,3}a_{3,1}+a_{1,3}a_{2,1}a_{3,2}-a_{1,2}a_{2,1}a_{3,3}-a_{1,1}a_{2,2}a_{3,1}-a_{1,1}a_{2,3}a_{3,2}
\end{align*}
\end{enumerate}
\subsubsection{Lemma}
Für alle $A,B\in\mathbb{K}^{n\times n}$ gilt
\renewcommand{\labelenumi}{(\alph{enumi})}
\begin{enumerate}
\item det$A=$det$A^T$
\item Entsteht $B$ durch eine Permutation $\sigma \in S_n$ der Spalten von $A$ (d.h. ist formal $B=(a_{\sigma (1)}\,a_{\sigma (2)},\cdots ,a_{\sigma (n)})$), oder der Zeilen von $A$ $\left(\text{d.h. formal} B=\begin{pmatrix}a^{\sigma (1)}\\ \vdots \\ a^{\sigma (n)}\end{pmatrix}\right)$, so gilt der$B=$sgn$\sigma$det$A$
\item Falls zwei Spalten der Zeilen von $A$ übereinstimmen, so ist det$A=0$.
\end{enumerate}
\underline{Beweis}: Es seien $A,B\in\mathbb{K}^{n\times n}$
\begin{enumerate}
\item Durch direktes Nachrechnen und \hyperref[4.1a]{$4.1a$} folgt 
\begin{align*}
\mathrm{det}A^T&\stackrel{(4,1c)}{=}\sum _{\sigma\in S_n}\mathrm{sgn}\sigma \prod _{i=1}^n a_{\sigma (i)i}\\
&=\sum _{\sigma \in S_n} \mathrm{sgn}\sigma ^{-1}\prod _{j=1}^n a_{j\sigma ^{-1}(j)}\\
&= \mathrm{det}A
\end{align*}
\item folgt ähnlich und (c) ist etwas involvierter.
\end{enumerate}
Eine zentrale Eigenschaft von Determinanten ist ihre Multiplikativität.  Allerdings ist sie in Dimensionen $n>1$ nicht additiv: Beispiel det$(I_n+I_n)=2^n$, aber det$I_n=1$.
\subsubsection{Satz (Multiplikativität der Determinante}
Es gilt $\label{4.1d}(4.1d)$det$(AB)=$det$A\cdot$det$B$ für alle $A,B\in\mathbb{K}^{n\times n}$.