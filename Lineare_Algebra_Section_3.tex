\section{Lineare Abbildungen}
Es seien $X,Y$ lineare Räume über dem selben Körper $\mathbb{K}$.
\subsection{Grundlagen}
\subsubsection{Definition (lineare Abbildung)}
Eine \underline{lineare Abbildung} $T: X\rightarrow Y$ erfüllt die Eigenschaft
\[\phantomsection\label{3.1a}(3.1a)\ T(\alpha _1x_1+\alpha _2x_2)=\alpha _1Tx_1+\alpha _2Tx_2\]
für alle $\alpha _1,\alpha _2\in\mathbb{K},x_1,x_2\in X$.  Für die Menge aller solchen linearen Abbildungen schreiben wir $L(X,Y)$.\\
Für lineare Abbildungen schreibt man $Tx:=T(x)$.
\subsubsection{Bemerkung}
\renewcommand{\labelenumi}{(\arabic{enumi})}
\begin{enumerate}
\item Für $T\in L(X,Y)$ ist $T0=0$
\item Die Menge $L(X,Y)$ ist ein Unterraum von $F(X,Y)$; wir kürzen ferner ab $L(X):=L(X,X)$.  Ist $Z$ ein weiterer linearer Raum und $T\in L(X,Y),S\in L(Y,Z)$, so ist auch die Komposition $S\circ T:X\rightarrow Z$ linear. $(L(X),\circ )$ ist eine Halbgruppe mit neutralem Element $id_x$.
\end{enumerate}
\subsubsection{Beispiel}
Die \underline{Nullabbildung} $0:X\rightarrow Y,\ 0x:=0\in Y$ ist linear, wie auch die identische Abbildung $id_x:X\rightarrow X$ aus \hyperref[identische]{Beispiel \ref*{identische}}
\subsubsection{Beispiel (affine Abbildungen)}
Eine Abbildung von $S:X\rightarrow Y$ heißt \underline{affin}, falls es $T\in L(X,Y)$ und $y\in Y$ derart gibt, dass $S(x)=Tx+y$.  $S$ ist genau dann linear , falls $y=0$.
\subsubsection{Beispiel (die Abbildung $T_A$)}
\label{3.1.5}
Die wichtigsten linearen Abbildungen dieser Vorlesung sind von der Form $T_A:\mathbb{K}^n\rightarrow \mathbb{K}^m,\ T_Ax=Ax$ mit $A\in\mathbb{K}^{m\times n}$.  Auch die Abbildung $\mathbb{K}^{m\times n}\rightarrow L(\mathbb{K}^n,\mathbb{K}^m),\ A\mapsto T_A$ ist linear.
\subsubsection{Beispiel}
\label{3.1.6}
\begin{enumerate}
\item Es sei $\Omega \not= \emptyset$ eine Menge und $x\in \Omega$.  Dann ist die \underline{Auswertung} $ev_x: F(\Omega ,X)\rightarrow X,\ ev_x(u):=u(x)$ linear.
\item Es sei $I\subseteq \mathbb{R}$ ein Intervall.  Dann ist die \underline{Differenziation} $D:C^1(I,\mathbb{R})\rightarrow C(I,\mathbb{R}),\ Du:=u'$ linear.
\item Mit einem Intervall $I\subseteq \mathbb{R}$, den fixen $t_0\in I$ und reellen Zahlen $a<b$ definieren auch nachfolgende Integrale lineare Abbildungen:
\[T_1:C([a,b],\mathbb{R})\rightarrow \mathbb{R}\ T_{1,u}:=\int_a^b u(s)ds\]
\[T_2:C(I,\mathbb{R})\rightarrow C^1(I,\mathbb{R}),\ (T_{2,u})(t):=\int_{t_0}^t u(s)ds\]
\end{enumerate}
\subsubsection{Beispiel (Vorwärts-Shift)}
Es sei $X$ ein linearer Raum und $\mathbb{I}\in\{\mathbb{N}_0,\mathbb{Z}\}$.  Bezeichnet dann $l(\mathbb{I})$ den linearen Raum aller Folgen $F(\mathbb{I},X)$, so ist der durch $(S\phi )_k:=\phi _{k+1}$ definierte \underline{Vorwärts-Shift} eine Abbildung $S\in L(l(\mathbb{I}))$.
\subsubsection{Definition (Kern, Bild, Rang)}
\label{KBR}
Ist $T\in L(X,Y)$, so bezeichnet $N(T):=\{x\in X:Tx=0\}$ den \underline{Kern}, $R(T):=TX$ das \underline{Bild} und rk$T:=$dim$R(T)$ den \underline{Rang} von $T$.  Eine Verbindung des Begriffes "`Rang einer Matrix"' (\hyperref[Mrang]{Definition \ref*{Mrang}}) und \hyperref[KBR]{Definition \ref*{KBR}} wird in \hyperref[3.3.8]{Satz \ref*{3.3.8}} hergestellt.
\subsubsection{Proposition}
Für jedes $T\in L(X,Y)$ ist der Kern $N(T)$ ein Unterraum von $X$.\\
Beweis: Übungsaufgabe.
\subsubsection{Satz}
\label{3.1.10}
Für jedes $T\in L(X,Y)$ gilt:
\renewcommand{\labelenumi}{(\alph{enumi})}
\begin{enumerate}
\item T ist genau dann injektiv, wenn $N(T)=\{0\}$
\item T ist genau dann surjektiv, wenn $R(T)=Y$
\end{enumerate}
\underline{Beweis}:
\begin{enumerate}
\item Die Abbildung $T$ ist genau dann nicht injektiv, wenn es $y\in Y$ und $x_1,x_2\in X$ derart gibt, dass $x_1\not= x_2$ und $Tx_1=y=Tx_2$.  Dies ist äquivalent zu $T(x_1-x_2)=0$, also $0 \not= x_1-x_1 \in N(T)$
\item ist genau die Definition von Surjektivität.
\end{enumerate}
\subsubsection{Beispiel}
\renewcommand{\labelenumi}{(\arabic{enumi})}
\begin{enumerate}
\item Die Auswertung $ev_x:F(\Omega ,X)\rightarrow X$ aus \hyperref[3.1.6]{Beispiel \ref*{3.1.6} (1)} hat den Kern $N(ev_x):=\{m\in F(\Omega ,X):u(x)=0\}$ und das Bild $R(ev_x)=X$, ein Urbild zu einem beliebigen $y\in X$ ist gerade die konstante $u(x)\equiv y$ auf $\Omega$.
\item Für die Nullabbildung $0\in L(X,Y)$ ist $N(0)=X$ und $R(0)=\{0\}$, für $X\not= \{0\}$ ist $0$ nicht injektiv.  Für $Y\not= \{0\}$ ist $0$ nicht surjektiv.
\item Mit einer Matrix $A\in \mathbb{K}^{m\times n}$ ist $T_A\in L(\mathbb{K}^n,\mathbb{K}^m)$ aus \hyperref[3.1.5]{Beispiel \ref*{3.1.5}} genau dann
\begin{itemize}
\item \underline{injektiv}, wenn die linear homogene Gleichung ($L_0$) nur die triviale Lösung hat.
\item \underline{surjektiv}, wenn es für jede Inhomogenität $b\in\mathbb{K}^m$ mindestens eine Lösung $x\in\mathbb{K}^n$ von ($L_b$) gibt.
\end{itemize}
\item Bei der Differenziation $D:C^1([a,b],\mathbb{R})\rightarrow C([a,b],\mathbb{R})$ aus \hyperref[3.1.6]{Beispiel \ref*{3.1.6} (2)} besteht der Kern $N(D)$ aus allen konstanten Funktionen.  Für das Bild $R(D)$ erhalten wir dagegen $C([a,b],\mathbb{R})$, denn für ein beliebiges $v\in C([a,b],\mathbb{R})$ gilt nach dem Hauptsatz der Differential- und Integralrechnung $Du=v$ mit $u(t):=v(a)+\int_a^t v(s)ds$.  Damit ist $D$ nicht injektiv, aber surjektiv.
\end{enumerate}
\subsubsection{Satz (Dimensionssatz)}
\label{dimension}
Für jede $T\in L(X,Y)$ mit dim$X<\infty$ gilt dim$N(T)+$dim$R(T)=$dim$X$.\\
\underline{Beweis}: Es sei $\{x_1,\cdots ,x_m\}$ eine Basis von $N(T)$ und $\{y_1,\cdots ,y_n\}$ eine Basis von $R(T)$.  Wir wählen $\hat{x}_1,\cdots ,\hat{x}_n \in X$ derart, dass $T\hat{x}_i=y_i,y\leq i\leq n$ gilt und weisen nach, dass $\mathcal{X}:=\{x_1,\cdots ,x_m,\hat{x}_1,\cdots ,\hat{x}_n\}$ eine Basis von $X$ ist.
\renewcommand{\labelenumi}{(\roman{enumi})}
\begin{enumerate}
\item \underline{$\mathcal{X}$ ist linear unabhängig}: dim$N(T)$+dim$R(T)$=dim$X$
\item \underline{$\mathcal{X}$ ist ein EZS}: dim$m$+dim$n$=dim$X$
\end{enumerate}
\subsubsection{Korollar}
Sei $T\in L(X,Y)$ ist $\{x_1,\cdots ,x_n\}$ eine Basis von $N(T)$ und $\{x_1,\cdots ,x_n,x_{n+1},\cdots ,x_d\}$ eine Basis von $X$ mit $n<d$.  Das Bild $R(T)$ hat folgende Basis:
\[\{Tx_{n+1},\cdots ,Tx_d\}\]
\underline{Beweis}:  Sei $d=$dim$X$,$n=$dim$N(T)$.  Nach \hyperref[dimension]{Dimensionssatz \ref*{dimension}} gilt:
dim$R(T)=d-n$.  Wir suchen $d-n$ linear unabhängige Vektoren in $R(T)$.  Die Vektoren $\{Tx_{n+1},\cdots ,Tx_d\}$ sind $d-n$ Vektoren in $R(T)$.  Wir zeigen , dass diese linear unabhängig sind. Hierzu gehen wir indirekt vor.  Wir nehmen an, dass $\{Tx_{n+1},\cdots ,Tx_d\}$ linear abhängig sind.\\
$\Rightarrow$ Es existiert ein Index $j^*,n<j^*\leq d$, so dass $Tx_{j^*}=\sum_{\substack{j=n+1\\ j\not= j^*}}^d \eta _jTx_j$.  Das heißt $\sum_{j=n+1}^d \eta _jTx_j=0$ mit $\eta _{j^*}=-1 \circledast\circledast$.  Aus der Linearität von $T$ folgt: $T(\sum_{j=n+1}^d \eta _jx_j)=0$, d.h. $\sum_{j=n+1}^d \eta _jx_j \in N(T)$.  Weil $\{ x_1,\cdots ,x_n\}$ Basis von $N(T)$ ist gilt: $\sum_{j=n+1}^d\eta _jx_j=\sum_{j=1}^n\eta _jx_j$ für geeignete $n_1,\cdots ,n_n\in\mathbb{K}$, also: $\sum_{j=1}^n\eta _jx_j-\sum_{j=n+1}^d\eta _jx_j=0 \circledast$\\
Weil nach Voraussetzung $x_1,\cdots x_d$ Basis von $X$ ist, folgt aus $\circledast$, dass $\eta _j=0\ \forall 1\leq j\leq d$ (Definition der linearen Unabhängigkeit).  Das ist ein Widerspruch zu $\circledast\circledast$.  Das heißt $\{Tx_{n+1},\cdots ,Tx_d\}$ sind linear unabhängig.
\subsubsection{Satz (Prinzip der linearen Fortsetzung)}
\label{lfortsetzung}
Sei $\{x_1,\cdots ,x_n\}$ ein Basis von $X$ und $\{\hat{y}_1,\cdots ,\hat{y}_n\} \in Y$.
\renewcommand{\labelenumi}{(\alph{enumi})}
\begin{enumerate}
\item Sind $T,S\in L(X,Y)$ zwei linearen Abbildungen mit $Tx_i=Sx_i\ \forall 1\leq i\leq n$.  Dann gilt $T=S$
\item Es existiert genau eine lineare Abbildung $T\in L(X,Y)$ mit $Tx_i=\hat{y}_i,\ \forall 1\leq i\leq n$
\end{enumerate}
\subsubsection{Bemerkung}
Für gegebenes $x=\sum_{k=1}^n\xi _kx_k,\ \xi _k\in \mathbb{K}\ \forall 1\leq k\leq n$, gilt $Tx=T(\sum_{k=1}^n\xi _kx_k)\stackrel{\hyperref[3.1a]{(3.1a)}}{=}\sum_{k=1}^n\xi _k Tx_k$.  Kenntnis der Koeffizienten $\xi _k$ und der Werte $Tx_i,1\leq i\leq n$ erlaubt uns den Wert von $Tx$ zu bestimmen.\\
\underline{Beweis} (\hyperref[lfortsetzung]{Satz \ref*{lfortsetzung}}):
\begin{enumerate}
\item Sei $Tx_i=Sx_i;\ \forall 1\leq i\leq n$.  Sei $x\in X$ mit $x=\sum_{i=1}^n\xi _ix_i,\ 1\leq i\leq n\ \xi _i \in \mathbb{K}$.
\[Tx=\sum_{i=1}^n \xi _iTx_i=\sum_{i=1}^n \xi _i Sx_i=S(\sum_{i=1}^n\xi _i x_i)=Sx\].
\item Wir definieren $T$ wie folgt.  Der Vektor $x$ habe die Darstellung $x=\sum_{i=1}^n \xi _ix_i$ $\xi _i\in\mathbb{K}$.  Wir definieren $Tx:=\sum_{i=1}^n \xi _i \hat{y}_i$.  Dann gilt $Tx_j=\sum_{j=1}^n \xi _{i,j} \hat{y}_i=\hat{y}_j$.  Zeige noch: $T$ ist linear.  Sei $z\in X$ dargestellt als $z=\sum_{i=1}^n\beta _ix_i$ und $\lambda \in \mathbb{K}$.\\
\underline{zeige}: $T(x+ z)=T(x)+ T(z)$ und $T(\lambda x)=\lambda T(x)$
\[T(x+z)=T(\sum_{i=1}^n \xi _i x_i +\sum_{i=1}^n \beta _ix_i) = T(\sum_{i=1}^n (\xi _i+\beta _i)x_i)\stackrel{def.}{=} \sum_{i=1}^n(\xi _i+\beta _i)\hat{y}_i=\sum_{i=1}^n\xi _I\hat{y}_i+\sum_{i=1}^n \beta _i \hat{y}_i=Tx+Tz\]
\[T(\lambda x)=\lambda Tx\ \mathrm{analog}\]
\end{enumerate}
\subsection{Isomorphismen}
\subsubsection{Definition}
Eine bijektive Abbildung $T\in L(X,Y)$ heißt Isomorphismus, und wir definieren $GL(X,Y)=\{T\in L(X,Y):T\text{ bijektiv}\}$.  Lineare Räume $X, Y$ werden als isomorph bezeichnet, wenn es einen Isomorphismus $T\in L(X,Y)$ gibt.  Schreibweise: $X\cong Y$.
\subsubsection{Bemerkung}
\begin{enumerate}
\item Wenn $Z$ ein weiterer $\mathbb{K}$-Vektorraum ist, und $T\in GL(X,Y)$ und $S\in GL(Y,Z)$, dann ist $S\circ T\in GL(X,Z)$. bildlich: $X\stackrel{T}{\rightarrow}Y\stackrel{S}{\rightarrow}Z$.  Wir schreiben $GL(X)$ für $GL(X,X)$.  Mit neutralem Element id$_x$ wird $GL(X)$ zu einer Gruppe, der sogenannten General Linear Group.  Achtung: $GL(X)$ ist kein Untervektorraum von L(X)!
\item Durch $A=\{(X,Y): X\text{ und } Y\text{ sind Isomorph}\}$ wird eine Äquivalenzrelation auf der Menge aller linearen Räume erklärt.
\end{enumerate}
\subsubsection{Beispiel (Transponierte)}
Die Abbildung $\circ ^T:K^{n\times m}\rightarrow K^{m\times n}$
\[ A=(a_{i,j})_{\substack{1\leq i\leq n\\ 1\leq j\leq m}} \mapsto A^T =(a_{j,i})_{\substack{1\leq i\leq n\\ 1\leq j\leq m}}\]
\begin{center}
(Zeilen und Spalten vertauschen!)
\end{center}
ist ein Isomorphismus.  Es gilt das Inverse des Transponieren ist das Transponieren selbst, d.h. $((A)^T)^T=A$ (für $n=m$).  Damit ist der Raum der $n$-Spalten isomorph zum Raum der $n$-Zeilen.
\subsubsection{Beispiel (Polynome)}
\renewcommand{\labelenumi}{(\arabic{enumi})}
\begin{enumerate}
\item Sei $\mathbb{K}\in \{\mathbb{Q},\mathbb{R},\mathbb{C}\}$ und $P_n(\mathbb{K})$ die Polynome von maximalem Grad $n\in \mathbb{N}_0$. $P_n(\mathbb{K})$ ist isomorph zu $\mathbb{K}^{n+1}$ via den Isomorphismus $T: P_n(\mathbb{K})\rightarrow \mathbb{K}^{n+1}$, $p\rightarrow (\alpha _0,\cdots ,\alpha _n)$, wobei $p(t)=\sum_{k=0}^n\alpha _kt^k$. Zum Beispiel: $p(t)=t^2+t\ p\mapsto (0,1,1, 0,\cdots ,0)$.
\item $l_{0,0}=\{(\alpha _k)_{k\in \mathbb{N}_0}:\exists n_0\ \forall n\geq n_0\ \alpha _n=0\}$ bezeichnet die Menge aller Folgen, die schließlich $0$ sind.\\
$l_{0,0}$ ist Isomorph zum Raum der Polynome $P(\mathbb{K})$ via den Isomorphismus $T\cdot C P(\mathbb{K})\rightarrow l_{0,0}$; $p\mapsto (\alpha _0,\cdots ,\alpha _n,0,\cdots, 0,\cdots ): p=\sum_{k=0}^n\alpha _kt^k$
\end{enumerate}
\subsubsection{Lemma}
\label{3.2.5}
Für $T\in L(X,Y)$ ist mit $\mathcal{S}\subseteq X$ auch $T\mathcal{S}$ linear abhängig (in $Y$).
\subsubsection{Satz}
\label{3.2.6}
Es sei $T\in GL(X,Y)$.  Eine Menge $\mathcal{S}\subseteq X$ ist genau dann linear abhängig, wenn $T\mathcal{S}\subseteq Y$ linear abhängig ist.
\subsubsection{Bemerkung}
Als logische Kontraposition erhalten wir, dass Isomorphismen linear unabhängige Mengen (oder Basen) auf ebensolche abbilden.\\
\underline{Beweis}:
\begin{itemize}
\item $(\Rightarrow )$ Folgt aus \hyperref[3.2.5]{Lemma \ref*{3.2.5}}
\item $(\Leftarrow )$ Nun sei $T\mathcal{S}\subseteq Y$ linear abhängig.  Wegen $T\in GL(X,Y)$ ist auch $T^{-1}(T\mathcal{S})$ nach \hyperref[3.2.5]{Lemma \ref*{3.2.5}} linear abhängig.
\end{itemize}
\subsubsection{Proposition}
\label{3.2.8}
Jeder $n$-dimensionale lineare Raum ist isomorph zu $\mathbb{K}^n,\ n\in\mathbb{N}_0$.\\
\underline{Beweis}: Es sei $X$ ein linearer Raum mit $m:=$dim$X$ und der Basis $\mathcal{X}=\{x_1,\cdots ,x_n\}$.  Wir definieren $T:\mathbb{K}^n\rightarrow X,\ \begin{pmatrix}\xi _1\\ \vdots \\ \xi _n\end{pmatrix}\mapsto \sum_{k=1}^n\xi _k x_k$.  Aufgrund der linearen Unabhängigkeit von $\mathcal{X}$ ist $N(T)=\{0\}$, nach \hyperref[3.1.10]{Satz \ref*{3.1.10}(a)} ist $T$ dann injektiv.  Da $\mathcal{X}$ ein EZS ist, muss $T$ auch surjektiv sein.
\subsubsection{Satz}
Endlich dimensionale lineare Räume $X,Y$ sind genau dann isomorph, wenn dim$X=$dim$Y$.\\
\underline{Beweis}
\begin{itemize}
\item $(\Leftarrow )$ Es sei $n:=$dim$X=$dim$Y$.  Nach \hyperref[3.2.8]{Proposition \ref*{3.2.8}} existieren Isomorphismen $\Phi :X\rightarrow\mathbb{K}^n,\Psi :Y\rightarrow \mathbb{K}$, womit $\Psi ^{-1} \circ \Phi \in GL(X,Y)$ ein Isomorphismus ist.
\item ($\Rightarrow$) Sei $T\in GL(X,Y)$, $\mathcal{X}=\{x_1,\cdots ,x_n\}$.  Dann ist $y_i:=Tx_i,\ 1\leq i\leq n$ nach \hyperref[3.2.6]{Satz \ref*{3.2.6}} Basis von $Y$.
\end{itemize}
\subsubsection{Satz}
\label{3.2.10}
Für jedes $T\in L(X,Y)$ zwischen linearen Räumen $X,Y$ mit dim$X=$dim$Y$ sind äquivalent:
\renewcommand{\labelenumi}{(\alph{enumi})}
\begin{enumerate}
\item $T$ ist ein Isomorphismus (d.h. $T\in GL(X,Y))$
\item $T$ ist injektiv
\item $T$ ist surjektiv
\end{enumerate}
\underline{Beweis}: Es ist nachzuweisen, dass Surjektivität und Injektivität äquivalent sind.  Es sei $T\in L(X,Y)$ injektiv.  Wegen \hyperref[3.1.10]{Satz \ref*{3.1.10}(a)} ist dies äquivalent zu $N(T)=\{0\}$.  Mit dem  \hyperref[dimension]{Dimensionssatz\ref*{dimension}} ist dann dim$R(T)$=dim$X-$dim$N(T)=$dim$Y$ und $T$ ist genau dann injektiv, wenn dim$R(T)$=dim$Y$, d.h. $R(T)=Y$ gilt.  Aufgrund von \hyperref[3.1.10]{Satz \ref*{3.1.10}(b)} folgt die Behauptung.
\subsection{Lineare Abbildungen und Matrizen}
$X,Y$ lineare Räume, $\mathcal{X}=\{x_1,\cdots ,x_n\},\ \mathcal{Y}:=\{y_1,\cdots ,y_m\}$.  Wir folgern aus \hyperref[lfortsetzung]{Satz \ref*{lfortsetzung}}, dass eine lineare Abbildung $T\in L(X,Y)$ durch die Bilder $Tx_i$ der Basisvektoren von $X$ bestimmt ist.  Sind etwa:\\
\[(3.3a)\ Tx_i = \sum_{j=1}^m a_{i,j} y_j\text{ für }1\leq i\leq n\]
und $x=\sum_{k=1}^n\xi _k x_k$ mit $Tx=\sum_{j=1}^m \eta _j y_j$, so resultiert $Tx=T(\sum_{k=1}^n\xi _k x_k)=\sum_{k=1}^n \xi _k Tx_k \stackrel{3.3a}{=} \sum_{k=1}^n\xi _k\sum_{j=1}^ma_{k,j}y_j=\sum_{i=1}^m(\sum_{k=1}^na_{i,k}\xi_k)y_i$.  Für die Koordinaten $\eta _i,\cdots ,\eta _m\in\mathbb{K}$ von $Tx$ bzgl. $\mathcal{Y}$ erhalten wir:\\
\[(3.3b)\ \eta _i=\sum_{k=1}^na_{i,k}\xi _k\text{ für }1\leq i\leq m\]
\subsubsection{Satz (darstellende Matrix)}
\label{Mdarstellend}
Jedes $T\in L(X,Y)$ wird eindeutig durch eine Matrix $T_\mathcal{X}^\mathcal{Y}\in\mathbb{K}^{m\times n}$ beschreiben, in deren $k-$ter Spalte gerade die Koordinaten von $Tx_k$ bzgl der Basis $\mathcal{Y}$ stehen.  Man nennt $T_\mathcal{X}^\mathcal{Y}$ die $T$ darstellende Matrix in den Basen $\mathcal{X}$ und $\mathcal{Y}$; im Fall $X=Y$ und $\mathcal{X}=\mathcal{Y}$ schreiben wir $T_\mathcal{X}:=T^\mathcal{X}_\mathcal{X}$.
\subsubsection{Bemerkung}
Wir versehen $X=\mathbb{K}^n$ und $Y=\mathbb{K}^m$ mit Standardbasen $\mathcal{E}_n$ bzw. $\mathcal{E}_m$ aus \hyperref[standardbasis]{Beispiel \ref*{standardbasis}}.  Für jede Abbildung $T\in L(\mathbb{K}^n,\mathbb{K}^m)$ mit darstellender Matrix \[T^{\mathcal{E}_m}_{\mathcal{E}_n}\text{ gilt dann }T=T_{T_{\mathcal{E}_n}^{\mathcal{E}_m}}\text{ (Erinnerung }T_Ax=Ax)\]
\subsubsection{Beispiel (Polynome)}
Es sei $n\in \mathbb{N}$ und $X=P_n(\mathbb{R})$ ausgestattet mit der monomialen Basis $\mathcal{M}_n:=\{m_0,\cdots ,m_n\}$ aus \hyperref[2.4.4]{Beispiel \ref*{2.4.4}}.  Als lineare Abbildung betrachten wir die Ableitung $D:P_n(\mathbb{R})\rightarrow P_n(\mathbb{R})$ mit den Bildern
\[D_{m_0}=0,\ D_{m_k}=km_{k-1}\text{ für alle }k\in \mathbb{N}\]
und erhalten aus \hyperref[Mdarstellend]{Satz \ref*{Mdarstellend}} die darstellende Matrix
\[ D_{\mathcal{M}_n}=\begin{pmatrix}0 & 1 & 0 & 0 &\cdots &0\\ 0 & 0& 2 & 0 &\cdots &0\\ 0& 0& 0& 3&\cdots &0\\ \vdots \\ 0 & 0& 0 &0 & \cdots & n-1\\ 0 & 0& 0 &0 & \cdots & 0\end{pmatrix}\]
\subsubsection{Proposition}
\label{3.3.4}
Zu jedem $A\in K^{m\times n}$ gibt es ein $T\in L(X,Y)$ mit $A=T_\mathcal{X}^\mathcal{Y}$.\\
\underline{Beweis}: Es sei $A\in \mathbb{K}^{m\times n}$.  Zu beliebigen $x\in X$ finden wir $\xi _i,\cdots ,\xi _n\in\mathbb{K}$ mit $x=\sum_{k=1}^n \xi _kx_k$.  Die gesuchte Abbildung $T\in L(X,Y)$ ist dann
\[Tx:=\sum_{i=1}^m\left(\sum_{j=1}^na_{i,j}\xi _j\right)y_i\]
\subsubsection{Korollar}
Für endlich dimensionale Räume $X,Y$ sind $L(X,Y)$ und $\mathbb{K}^{\mathrm{dim}Y\times \mathrm{dim}X}$ isomorph.  Insbesondere ist dim$L(X,Y)=$dim$X\cdot$dim$Y$.\\
\underline{Beweis}: Es sei $m=$dim$Y$,$n=$dim$X$.  Wir verwenden die lineare Abbildung $\Phi :L(X,Y)\rightarrow \mathbb{K}^{\mathrm{dim}Y\times \mathrm{dim}X}$,$\Phi (T)=T_\mathcal{X}^\mathcal{Y}$ die in \hyperref[3.3.4]{Proposition \ref*{3.3.4}} konstruiert wurde.
\subsubsection{Satz}
Es sei $Z$ ein weiterer linearer Raum mit Basis $\mathcal{Z}$.  Für $T\in L(X,Y),\ S\in L(Y,Z)$ ist
\[ (S\circ T)_\mathcal{X}^\mathcal{Z}=S_\mathcal{Y}^\mathcal{Z}T_\mathcal{X}^\mathcal{Y}\]
\subsubsection{Bemerkung}
Für $A\in\mathbb{K}^{l\times m},\ B\in\mathbb{K}^{m\times n}$ gilt
\[(3.3c)\ T_A\circ T_b=T_{AB}\]
\addtocounter{subsubsection}{-6}
\subsubsection{Die Abbildung $T_A$}
\label{TA}
In diesem Abschnitt sei $A\in\mathbb{K}^{m\times n}$ und $T_A\in L(\mathbb{K}^n,\mathbb{K}^m)$ mit $T_Ax:=Ax$.
\addtocounter{subsubsection}{5}
\subsubsection{Satz}
\label{3.3.8}
Die Ränge von $A\in\mathbb{K}^{m\times n}$ und $T_A\in L(\mathbb{K}^n,\mathbb{K}^m)$ stimmen überein: rk$T_A=$rk$A$.
\subsubsection{Bemerkung}
\label{3.3.9}
Mit den Spalten $a_1,\cdots a_n\in \mathbb{K}^m$ von $A$ gilt $R(T_A)=$span$\{a_1,\cdots ,a_n\}$, weshalb insbesondere $R(T_A)$ und der Spaltenraum von $A$ gleiche Dimension haben.  Andererseits war rk$A$ nach \hyperref[Mrang]{Definition \ref*{Mrang}} die Dimension des Zeilenraumes von $A$.  Daher wird \hyperref[3.3.8]{Satz \ref*{3.3.8}} auch formuliert als: "`Spaltenrang=Zeilenrang"'.\\
\underline{Beweis}: Zunächst merken wir an, dass $N(T_A)$ mit dem Lösungsraum $L_0\subseteq \mathbb{K}^n$ einer homogenen Gleichung $(L_0)$ übereinstimmt.  Nach \hyperref[2.6.3]{Proposition \ref*{2.6.3}} gilt also dim$N(T)=n-$rk$A$ Andererseits liefert der \hyperref[dimension]{Dimensionssatz \ref*{dimension}}, dass $n=\mathrm{dim}N(T_A)+\mathrm{dim}N(T_A)=\mathrm{dim}N(T_A)+\mathrm{rk}T_A$ und folglich $\mathrm{rk}T_A=\mathrm{rk}A$.\\\\
Nun sei $A\in\mathbb{K}^{n\times n}$ quadratisch mit rk$A=n$.  Dies ist mit \hyperref[3.2.10]{Satz \ref*{3.2.10}} äquivalent dazu, dass $T_A\in L(\mathbb{K}^n)$ ein Isomorphismus ist.  Wir interessieren uns nun für die simultane Lösbarkeit von
$Ax=e;\text{ für alle } 1\leq i\leq n$ welche wir mittels der augmentierten Matrix ($A,e_1,\cdots e_n)=(A,I_n)\in \mathbb{K}^{n\times (2n)}$ notieren.  Vermöge des Gauß-Verfahrens lässt sich $(A,I_n)$ auf die Form $(I_n,B)$ mit einem $B\in\mathbb{K}^{n\times n}$ bringen; nach Konstruktion ist $A\cdot B=I_n$.
\subsubsection{Definition (inverse Matrix)}
Eine Matrix $A\in\mathbb{K}^{n\times n}$ heißt \underline{invertierbar}, falls ein $B\in\mathbb{K}^{n\times n}$ mit der Eigenschaft $A\cdot B=I_n$ existiert.  Man nennt $B$ die \underline{Inverse} von $A$ und schreibt $A^{-1}:=B$.
\subsubsection{Beispiel}
Wir wollen die Matrix $A:=\begin{pmatrix}1&2&3\\2&3&4\\3&4&6\end{pmatrix}$ invertieren.  Nach obigen Schema
\[\begin{array}{ccc|ccc}1&2&3&1&0&0\\2&3&4&0&1&0\\3&4&6&0&0&1\end{array}\Leftrightarrow \begin{array}{ccc|ccc}1&2&3&1&0&0\\0&1&2&2&-1&0\\0&2&3&3&0&-1\end{array}\Leftrightarrow \begin{array}{ccc|ccc}1&2&3&1&0&0\\0&1&2&2&-1&0\\0&0&1&1&-2&1\end{array}\Leftrightarrow \begin{array}{ccc|ccc}1&2&0&-2&6&-3\\0&1&0&0&3&-2\\0&0&1&1&-2&1\end{array}\]
\[\Leftrightarrow \begin{array}{ccc|ccc}1&0&0&-2&0&1\\0&1&0&0&3&-2\\0&0&1&1&-2&1\end{array}\]
und erhalten $A^{-1}=\begin{pmatrix}-2&0&1\\0&3&-2\\1&-2&1\end{pmatrix}$.\\
Eine testweise Multiplikation ergibt $AA^{-1}=A^{-1}A=I_3$.
\subsubsection{Korollar}
\label{3.3.12}
Die inverse Matrix $A^{-1}\in\mathbb{K}^{n\times n}$ ist eindeutig bestimmt mit $A^{-1}A=I_n$.  Ferner ist $A$ genau dann invertierbar, wenn $T_A\in GL(\mathbb{K}^n)$ ist.
\subsubsection{Definition (regulär, singulär)}
\label{regulaer}
Eine Matrix $A\in\mathbb{K}^{n\times n}$ heißt \underline{regulär}, falls rk$A=n$ gilt, andernfalls nennen wir sie \underline{singulär}.
\subsubsection{Satz (Charakterisierung regulärer Matrizen)}
\label{3.3.14}
Folgende Aussagen sind äquivalent für jedes $A\in\mathbb{K}^{n\times n}$
\begin{enumerate}
\item $T_A\in GL(\mathbb{K}^n)$
\item A ist regulär
\item Die Zeilen von $A$ sind linear unabhängig
\item Die Spalten von $A$ sind linear unabhängig
\item Die homogene Gleichung ($L_0$) hat nur die triviale Lösung
\item Für jedes $b\in\mathbb{K}^n$ ist ($L_0$) eindeutig lösbar.
\end{enumerate}
\underline{Beweis}: Die Äquivalenz von (b) und (c) ist \hyperref[regulaer]{Definition \ref*{regulaer}}.  Aufgrund von \hyperref[3.3.8]{Satz \ref*{3.3.8}} und \hyperref[3.3.9]{Bemerkung \ref*{3.3.9}} sind auch (c) und (d) gleichwertig.\\
(d) $\Rightarrow$ (e) resultiert aus \hyperref[unabhaengig]{Definition \ref*{unabhaengig}} der linearen Unabhängigkeit.\\
(e) $\Rightarrow$ (f) ergibt sich aus \hyperref[1.4.9]{Satz \ref*{1.4.9}(a)}.\\
(f) $\Rightarrow$ (a) Nach unserer Voraussetzung (f) ist $T_A^{-1}(\{b\})$ für jedes $b\in \mathbb{K}^n$ einpunktig.  Also ist $T_A\in L(\mathbb{K}^n)$ bijektiv.\\
(a) $\Rightarrow$ (b) Übung!\\
Zum Abschluss des Unterabschnittes beschäftigen wir uns mit linear inhomogenen Gleichungen: Ihre \underline{augmentierte Koeffizientenmatrix} $(A,b)\in\mathbb{K}^{m\times (n+1)}$ definieren wir durch
\[(A,b)=\begin{pmatrix}a_{1,1}& \cdots &a_{1,n} & b_1\\ \vdots & & \vdots & \vdots\\ a_{m,1}&\cdots &a_{m,n}&b_m\end{pmatrix}\]
\subsubsection{Satz}
Es sei $b\in\mathbb{K}^m$.  Eine inhomogene Gleichung $(L_b)$ hat genau dann eine Lösung, wenn gilt
\[\mathrm{rk}A=\mathrm{rk}(A,b)\]
\underline{Beweis}: Übungsaufgabe.
\addtocounter{subsubsection}{-13}
\subsubsection{Basiswechsel}
\label{basiswechsel}
Auf einem endlich dimensionalen Raum $X$ seien die Basen $\mathcal{X}:=\{x_1,\cdots ,x_n\}$ und $\mathcal{X}':=\{x'_1,\cdots ,x'_n\}$ gegeben.  Wie verhält sich die Darstellungsmatrix von $T\in L(X)$ wenn man von $\mathcal{X}$ auf die Basis $\mathcal{X}'$ übergeht?
\begin{itemize}
\item Zunächst lassen sich die Elemente von $\mathcal{X}'$ darstellen durch die Elemente von $\mathcal{X}$
\[(3.3e)\ X'_j=\sum _{i=1}^ns_{ij}x_i\text{ für alle } 1\leq j\leq n\]
Hieraus wird die sogenannte \underline{Basiswechselmatrix} $S=(s_{ij})_{1\leq i,j\leq n}$ gebildet, welche den Übergang von $\mathcal{X}$ nach $\mathcal{X}'$ liefert:
\begin{center}
Spalten von S = Koordinatenvektoren der "`neuen"' Basisvektoren
\end{center}
\item Umgekehrt lassen sich die $x_j$ durch die $x'_i$ ausdrücken.  Aus $x_j=\sum _{i=1}^ns_{ij}x'_i$ erhalten wir $x_j=\sum _{i=1}^ns_{ij}(\sum _{k=1}^n s_{ki}x_k)=\sum _{k=1}^n(\sum _{i=1}^n s_{ki}s'_{ij})x_k$. Der Ausdruck in der letzten Klammer ist genau das $(k,j)$-te Element von $SS'$.  Wegen der linearen Unabhängigkeit von $\mathcal{X}$ folgern wir, dass $SS'=I_n$ und somit $S'=S^{-1}$ sein muss.  Schließlich definiert man auch jede invertierbare Matrix $S$ einen Basiswechsel gemäß $(3.3e)$ bzw. $x'_j=Sx_j$.
\end{itemize}
\addtocounter{subsubsection}{12}
\subsubsection{Satz}
\label{3.3.16}
Ist $S\in \mathbb{K}^{n\times n}$ die Basiswechselmatrix zwischen $\mathcal{X}$ und $\mathcal{X}'$, so gilt $T_{\mathcal{X}'}=S^{-1}T_\mathcal{X}S$, für alle $T\in L(X)$.
\subsubsection{Definition (Ähnlichkeit)}
Zwei Matrizen $A,B\in \mathbb{K}^{n\times n}$ heißen \underline{ähnlich}, falls es eine reguläre Matrix $S\in\mathbb{K}^{n\times n}$ derart gibt, dass $B=S^{-1}AS$.
\subsubsection{Bemerkung}
\begin{enumerate}
\renewcommand{\labelenumi}{(\arabic{enumi})}
\item Laut \hyperref[3.3.16]{Satz \ref*{3.3.16}} sind Darstellungsmatrizen bezüglich verschiedener Basen ähnlich vermöge der Basiswechselmatrix
\item Die Ähnlichkeit von Matrizen definiert eine Äquivalenzrelation auf $\mathbb{K}^{n\times n}$.
\end{enumerate}
\underline{Ziel}: Finde "`einfache"' Repräsentanten.